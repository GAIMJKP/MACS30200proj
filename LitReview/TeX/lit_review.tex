\documentclass[11pt, oneside]{article}   	
\usepackage{geometry}                		% See geometry.pdf to learn the layout options. There are lots.
\geometry{letterpaper}                   		% ... or a4paper or a5paper or ... 
%\usepackage[parfill]{parskip}    		% Activate to begin paragraphs with an empty line rather than an indent
\usepackage{graphicx}				% Use pdf, png, jpg, or eps§ with pdflatex; use eps in DVI mode
								% TeX will automatically convert eps --> pdf in pdflatex		
\usepackage{amssymb}

\usepackage{biblatex}	%references. Better than natbib or bibtex
\addbibresource{references.bib}

\usepackage{setspace}

\begin{document}

\begin{titlepage}
\title{'Using Uber Movement to assess how Social Events affect City Traffic': Literature Review \thanks{working title}}

\author{Laurence Warner\thanks{lpwarner@uchicago.edu}}
\date{}
\maketitle

\end{titlepage}

\begin{spacing}{2}
\section{Introduction}

I know that I want to use Uber Movement data to analyse the social determinants of traffic flows. But my exact analysis is still unclear. So I will try to cover breadth rather than depth in this lit review. I will organize my discussion by topic.

\section{Car studies}

\subsection{Taxis}

\autocite{turner2007axial} is an influential paper in the new space of transport network analysis. I will build upon the methodology of treating streets as a network.

\autocite{liu2015revealing} sets up a methodology of analysing the structure of a city. They convert the city of Shanghai into a graph to allow for network analysis.

\autocite{zheng2011urban} One of the first papers to analyse large-scale data on taxi movement: 30,000 taxicabs in Beijing. Uber Movement will allow me to scale up this analysis. 

\subsection{Uber Movement}

There now exists traffic data on 5 US cities. 

The only paper making use of this dataset so far is \autocite{pearson11traffic}. Using R packages, they do a bunch of interesting modeling of the data. I am in touch with the authors about how to port their analysis to my favored language of Python. 

Most relevant to me is the use of temporal analysis to compare traffic flows at different times. 

I will build upon the work, by undertaking work suggested in 'Section 6: Future Work'. This is to use a larger temporal data set to compare flows between different dates. Specifically, if we are interested in the effect of the Christmas holiday, we could compare December 25th to January 25th. 

\section{Social determinants of traffic}

\autocite{10.2307/1911814} Early paper framing the social importance of traffic. Conceptualises traffic as a public bad which is overconsumed due to the negative externality imposed upon other drivers. I am inspired by this to investigate how much drivers respond to special events, and how much private individuals decrease their driving at times of high demand.

\autocite{10.1007/3-540-48238-5_28} The first major paper to treat drivers as social beings and investigate the social causes of traffic.

In my paper this class, I will probably just choose one of the following topics to analyze, with a comparison between two cities.

\subsection{Transport}

When there is a public transport shutdown, how many displaced passengers move onto the road. How elastic is the decision to drive to the amount of traffic on the road?

Consider this Medium article on the London Bridge closure affecting other traffic: https://medium.com/uber-movement/examining-the-impact-of-the-london-tower-bridge-closure-5b7626e44915?lang=en-US

\subsection{Weather}

Pearson et al. recommend analysis seasonal trends. This paper \autocite{KILPELAINEN2007288}  used a survey of drivers to analyze how weather and weather forecasts affect driving behavior.
I propose to investigate the following hypothesis: average travel time is higher in winter.

\subsection{Holidays/Special Events}

\autocite{cools2007investigating} uses the Box-Tiao modeling approach to show that traffic counts are significantly lower during holiday periods. I hope to replicate this approach on my large-scale data.

\newpage
\printbibliography

\end{spacing}
\end{document}  `